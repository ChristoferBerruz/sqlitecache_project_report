
\section{Results and Discussion}
% Describe evaluation methodology and significant results in the evaluation section
% Evaluate the selected approach and analyze why the selected approach is good?
%   Provide an intuitive description of the algorithms, their correctness and their complexity
The simulation was run
on a Dell Laptop with a 11th Gen Intel(R) Core(TM) i7-11800H processor (8 cores, 16 threads)
running at 2.30GHz, 64.0 GB of installed RAM,
a 64-bit operating system with an x64-based processor
using Python3.10 and pytest as the driver.
We selected Pytest because the test-isolation generalizes
well for simulations. Therefore, simulations
are `tests' marked with the \texttt{@pytest.mark.simulation}
marker. Because running a simulation is slow,
these tests are also marked as \textit{slow}.

We let $P = 100$.
In our platform, all uncompressed and unencrypted
values in $D$ occupy $b = 18$ bytes.
Therefore, the cache size $P' = Q' = 1800$ bytes.
Furthermore, no compression and no encryption
was enabled.

The simulation was run as a sustained load,
as described in Algorithm~\ref{alg:simulation}.
This means that it runs until completion
for
$N$ requests. For example, in $N=1000$
we generate 1000 data points for both
the hit and miss rates.