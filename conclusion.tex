\section{Conclusion}
% Conclusions, lessons learned possible improvements, etc.
This project accomplished two goals. First,
we successfully implemented as single-threaded,
persistent, and high capacity key-value cache using
SQLite as a Python package: \sqlitecache.

Furthermore,
we implemented three eviction policies:
LRU, LFU, and Hybrid. Our hybrid eviction
policy was first introduced in~\cite{shah2023ImprovedCacheEviction}.
Additionally, all the functional features of our project
were functionally tested using \texttt{pytest}.
Our work with \sqlitecache~shows that it is possible to combine
relational databases and the filesystem, to effectively
take advantage of secondary storage for applications instead
of in-memory solutions. Our project is useful for workloads
where cache behavior is needed when handling
vast amounts of data.

Second, we successfully replicated the findings in~\cite{shah2023ImprovedCacheEviction}.
We used our persistent cache classes
to run a simulation to compare the hit rates and miss rates
among all eviction policies. The simulation
replicated unpredictable workloads by randomly inserting
and looking up elements drawn from a set of 500 values.
Our simulation demonstrated that the Hybrid policy
produces the highest hit rate, as well
as the lowest miss rate, among all eviction policies.
These findings indicate that the policy introduced in~\cite{shah2023ImprovedCacheEviction}
can be successfully applied to persistent cache strategies
with a high degree of adaptability to unpredictable workloads.
One contribution from our project was the introduction
of a more \textit{well-defined} simulation.

By creating \sqlitecache~, we gained a deeper insight
as to how SQLite works. The \texttt{ON CONFLICT DO}~clause
was useful when dealing with settings and recovery.
Triggers were practical to accurately keep track
of the size of the cache. We believe the biggest
lesson we learned is designing a cache \textbf{that works}
by using
a set of relations. This task forced us to think
not only about how to structure data, but also how to
reuse SQL databases for building useful data structures.
Finally, by
running a simulation with cache objects
with an I/O overhead, we learned the benefits
of a designing a simulation with parallelism in mind
and distributing the workload across all
available CPUs.

Although the results of our project were satisfactory,
there are several areas of improvements and expansions.

First, it will be interesting to create a multi-threaded and
distributed version of \sqlitecache.
Second, one area of expansion is to support different
serialization mechanisms besides pickle, such as JSON or
MessagePack.
Currently, our project only support storing pickle-serializable
values inside the cache. However, pickle presents
several security risks and portability issues~\cite{arjancodesPickle}.
Third, this project can be enhanced by adding
several other eviction policies including
Most Recently Used (MRU), Random Replacement (RR),
and First In-First Out (FIFO). Fourth,
one interesting research area will be to run
the simulation again with more eviction policies.
Finally, during our simulations, we hardcoded
the values for the counter threshold $T$ and TTL $\tau$
used by LFUCacheTTL and HybridCache. It is worth
experiment with different values for $(T, \tau)$,
either through grid search or a bayesian optimization,
and see their impact on hit rate and miss rate.
However, we acknowledge that performing parametrized
simulations or increasing the value of $N$ requests
might be challenging considering its runtime.